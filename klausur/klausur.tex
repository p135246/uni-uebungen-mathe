% ====
% Exam
% ====

\documentclass[a4paper,ngerman]{article}

% Load external packages

\usepackage{subfiles}
\usepackage[utf8]{inputenc}
\usepackage[ngerman]{babel}
\usepackage[top=3cm, bottom=3cm, left=3.5cm, right=3.5cm]{geometry}

% Load custom packages

\providecommand{\packages}{../latex_packages}
\usepackage{\packages/my_lehre_paketten}
\usepackage{\packages/my_lehre}

% Load exercises

% ================================
% Statements of all exam exercises
% ================================
%
% Use \PunkteL{} and \PunkteR{} to set the number of points
% Use \ZeitR{} and \ZeitL{} to assign a time estimate for solution 
%
% Example: Klausur, Aufgabe 1
%
% \newcommand{\KI}{\PunkteL{5}\ZeitR{20 Min}
%       Berechne 2+2.
% }


% Set heeaders and footers

\usepackage{fancyhdr}
\setlength{\headheight}{15.2pt}
\fancypagestyle{plain}{
	\fancyhf{}
	\fancyfoot[C]{\thepage}
	\renewcommand{\headrulewidth}{0pt}
	\renewcommand{\footrulewidth}{0pt}}
\fancyhf[HL]{
	% Klausur zur Analysis 2, 25.02.2017
	}
\fancyhf[HR]{
	Name: \hphantom{Max Mustermann von Musterwald} 
	% Matrikelnr.: \hphantom{1234567}
	}

% Formating

\newcommand{\NeueSeite}{\clearpage\thispagestyle{plain}\mbox{}\clearpage}
\renewcommand{\PunkteL}[1]{}

\begin{document}
%
% ==========
% Title page
% ==========
%
% \pagestyle{fancy}
% \thispagestyle{plain}
% %
% \AnaIIWSXVI{Klausur zur Analysis 2 \\[1ex] \normalsize{\normalfont 25. Februar 2017}}
% %
% Vorname:\ \hrulefill \\[\baselineskip]
% Nachname:\ \hrulefill \\[\baselineskip]
% Studienfach:\ \hrulefill \\[\baselineskip]
% Matrikelnummer:\ \hrulefill \\[\baselineskip]
% Kennwort (max. 8 Zeichen):\ \hrulefill \\[\baselineskip]
% $\square$ \quad \parbox[t]{13cm}{Ich stimme zu, dass mein Kennwort zusammen mit meiner Punktzahl oder Note im Digicampus veröffentlicht werden darf.}\\[1.5\baselineskip]
% Unterschrift:\ \hrulefill
% %
% \vspace{1\baselineskip}
%%
% \begin{center}
% \begin{tabu} to \textwidth {|*{8}{>{\centering}X[m]|}|>{\centering}X[1.5,m,c]|}\hline 
%  \rowfont{\large\bfseries}  1 & 2 & 3 & 4 & 5 & 6 & 7 & 8 & Gesamt \\ \hline\hline
%   \rule[-1.3em]{0pt}{3em}   &  &  &   &   &  & & &  \\ \hline
%   \rule[-1.3em]{0pt}{3em} &   &   &   &   &   & & & \\ \hline
% \end{tabu}
% \end{center}
%
% \vfill
% %
% \noindent {\bf\large Beachten Sie:}
% \begin{itemize}
% 	\item Die Bearbeitungszeit beträgt \textbf{180 Minuten}. 
% 	Bei jeder Aufgabe finden Sie Zeitangaben. Diese können Sie bei Ihrer Zeiteinteilung berücksichtigen.
% 		\item Es gibt \textbf{8 Aufgaben}. Überprüfen Sie, ob Ihre Klausur aus \textbf{20 Seiten} besteht. Das letzte Blatt ist ein Zusatzblatt. \textbf{Schreiben Sie nicht mit Bleistift oder Rotstift}.
% 	\item \textbf{Alle Behauptungen, Berechnungen, Resultate u.Ä. müssen begründet werden}. 
% 		Sie dürfen die Resultate aus der Vorlesung und den Übungen verwenden, 
% 		müssen sie aber klar anführen. Wird aber explizit nach einem Beweis aus der Vorlesung gefragt, so ist ein vollständiger Beweis anzugeben.
% 	\item \textbf{Ihre Antworten sollen eindeutig sein.}
% 		Falls Sie erkennen, dass Ihre ursprünglichen Überlegungen falsch waren, kennzeichnen Sie deutlich, was zu berücksichtigen ist.
% 	\item \textbf{Es sind keine Hilfsmittel erlaubt.} Jeder Versuch, sie trotzdem zu 
% 		verwenden, führt zum automatischen Nichtbestehen.
% 	\item \textbf{Merken Sie sich das geschriebene Kennwort}. Unter diesem werden Sie Ihre Note im 	Digicampus finden.
% 	%\item \textbf{Bei Aufgabe 8 ist deutlich anzugeben, ob die Aussage wahr oder falsch ist.} Sollten wir bei einer Teilaufgabe kein wahr oder falsch finden, können Sie in dieser Teilaufgabe nicht die volle Punktzahl erreichen.
% \end{itemize}
% %
% \vfill
% \begin{center}
% \textbf{Das ganze Analysis 2 Team wünscht Ihnen viel Erfolg.}
% \end{center}
%
%\NeueSeite
%
% ========
% Problems
% ========
%
\begin{Prob}
% \KAI
\end{Prob}
%
\NeueSeite
%
% ...
%
% =======================================
% A couple of empty pages for computations
% ========================================
%
\NeueSeite
\clearpage\mbox{}\clearpage
\NeueSeite
%
% ...
%
\end{document}
